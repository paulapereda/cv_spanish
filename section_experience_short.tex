% YAAC Another Awesome CV LaTeX Template
%
% This template has been downloaded from:
% https://github.com/darwiin/yaac-another-awesome-cv
%
% Author:
% Christophe Roger
%
% Template license:
% CC BY-SA 4.0 (https://creativecommons.org/licenses/by-sa/4.0/)
%Section: Work Experience at the top
\sectionTitle{Experiencia}{\faSuitcase}
%\renewcommand{\labelitemi}{$\bullet$}
\begin{experiences}
  \experience
    {Actualidad}   {Técnica en Evaluación y Monitoreo}{Transforma Uruguay} {Uruguay}
    {Marzo 2018} {
                      \begin{itemize}
                        \item Procesamiento y análisis de datos administrativos de programas y secundarios de diversas fuentes de información nacionales vinculadas al área productiva.                    
                        \item Elaboración de informes y documentos técnicos de monitoreo y evaluación de políticas productivas. 
                        
                         \item Desarrollo de capacidades de buenas prácticas en evaluación y monitoreo para agencias y ministerios vinculados.
                              
                        \item Desarrollo de aplicaciones para la visualización de información y la toma de decisiones.
                      \end{itemize}
                    }
                    {}
  \emptySeparator
  \experience
    {Marzo 2018} {Analista de Información Junior}{TATA, MultiAhorro, BAS}{Uruguay}
    {Diciembre 2017}    {
                      \begin{itemize}
                        
                     \item Responsable de gestionar la operativa del programa de fidelidad.
                 
                     \item Responsable de control de fraude.
                     \item Atención de cliente interno y externo.
                     \item Procesamiento y análisis de información del programa de fidelidad para la elaboración de estudios de mercado.  

                      \end{itemize}
                    }
                    {}
\emptySeparator
  \experience
    {Actualidad} {Co-organizadora}{R-Ladies Montevideo}{Uruguay}
    {Diciembre 2017}    {
                      \begin{itemize}
                        
                     \item R-Ladies Montevideo es parte de una organización mundial para promover la diversidad de género en la comunidad R (lenguaje de programación).


                      \end{itemize}
                    }
                    {}
\emptySeparator
  \experience
    {Actualidad} {Co-organizadora}{Grupo de Usuarios de R en  Uruguay (GURU::mvd)}{Uruguay}
    {Setiembre 2018}    {
                      \begin{itemize}
                        
                     \item El objetivo del grupo es tener un espacio de intercambio entre personas usuarias e interesadas en R de cualquier nivel, donde se puedan compartir experiencias, desafíos y se pueda aportar al desarrollo de la comunidad de R en Uruguay en un ambiente amigable.


                      \end{itemize}
                    }
                    {}

    
\end{experiences}
